\documentclass{ctexbook}

\usepackage[
pdftitle={三国演义},
pdfauthor={罗贯中},
pdfcreator={XeLaTeX},
colorlinks=true,
]{hyperref}

% papersize参数可以设置pdf尺寸
\usepackage[papersize={9.2cm,12.3cm},
left=0.5cm,
right=0.5cm,
%top=0.5cm,
bottom=0.5cm
]{geometry}

\ctexset{
chapter = {
name = {第,回},
number = \chinese{chapter}
},
chapter/format = \small\bfseries\centering
}

\usepackage{listings}
\lstset{basicstyle={\normalfont\sffamily},
	breaklines=true
}


\title{三国演义}
\author{罗贯中}
\date{明朝}

\begin{document}

\maketitle

\tableofcontents

\pagestyle{headings}

\chapter{宴桃园豪杰三结义 \ 斩黄巾英雄首立功}
\begin{quotation}
滚滚长江东逝水,浪花淘尽英雄。是非成败转头空。青山依旧在,几度夕阳红。白发渔樵江渚上,惯看秋月春风。一壶浊酒喜相逢。古今多少事,都付笑谈中。

\begin{flushright}
——调寄《临江仙》
\end{flushright}

\end{quotation}


话说天下大势,分久必合,合久必分。周末七国分争,并入于秦。及秦灭之后,楚、汉
分争,又并入于汉。汉朝自高祖斩白蛇而起义,一统天下,后来光武中兴,传至献帝,遂分
为三国。推其致乱之由,殆始于桓、灵二帝。桓帝禁锢善类,崇信宦官。及桓帝崩,灵帝即
位,大将军窦武、太傅陈蕃共相辅佐。时有宦官曹节等弄权,窦武、陈蕃谋诛之,机事不
密,反为所害,中涓自此愈横。

建宁二年四月望日,帝御温德殿。方升座,殿角狂风骤起。只见一条大青蛇,从梁上飞
将下来,蟠于椅上。帝惊倒,左右急救入宫,百官俱奔避。须臾,蛇不见了。忽然大雷大
雨,加以冰雹,落到半夜方止,坏却房屋无数。建宁四年二月,洛阳地震;又海水泛溢,沿
海居民,尽被大浪卷入海中。光和元年,雌鸡化雄。六月朔,黑气十余丈,飞入温德殿中。
秋七月,有虹现于玉堂;五原山岸,尽皆崩裂。种种不祥,非止一端。帝下诏问群臣以灾异
之由,议郎蔡邕上疏,以为蜺堕鸡化,乃妇寺干政之所致,言颇切直。帝览奏叹息,因起更
衣。曹节在后窃视,悉宣告左右;遂以他事陷邕于罪,放归田里。后张让、赵忠、封谞、段
珪、曹节、侯览、蹇硕、程旷、夏恽、郭胜十人朋比为奸,号为“十常侍”。帝尊信张让,
呼为“阿父”。朝政日非,以致天下人心思乱,盗贼蜂起。

 时巨鹿郡有兄弟三人,一名张角,一名张宝,一名张梁。那张角本是个不第秀才,因入
山采药,遇一老人,碧眼童颜,手执藜杖,唤角至一洞中,以天书三卷授之,曰:“此名
《太平要术》,汝得之,当代天宣化,普救世人;若萌异心,必获恶报。”角拜问姓名。老
人曰:“吾乃南华老仙也。”言讫,化阵清风而去。角得此书,晓夜攻习,能呼风唤雨,号
为“太平道人”。中平元年正月内,疫气流行,张角散施符水,为人治病,自称“大贤良
师”。角有徒弟五百余人,云游四方,皆能书符念咒。次后徒众日多,角乃立三十六方,大
方万余人,小方六七千,各立渠帅,称为将军;讹言:“苍天已死,黄天当立;岁在甲子,
天下大吉。”令人各以白土书“甲子”二字于家中大门上。青、幽、徐、冀、荆、扬、兖、
豫八州之人,家家侍奉大贤良师张角名字。角遣其党马元义,暗赍金帛,结交中涓封谞,以
为内应。角与二弟商议曰:“至难得者,民心也。今民心已顺,若不乘势取天下,诚为可
惜。”遂一面私造黄旗,约期举事;一面使弟子唐周,驰书报封谞。唐周乃径赴省中告变。
帝召大将军何进调兵擒马元义,斩之;次收封谞等一干人下狱。张角闻知事露,星夜举兵,
自称“天公将军”,张宝称“地公将军”,张梁称“人公将军”。申言于众曰:“今汉运将
终,大圣人出。汝等皆宜顺天从正,以乐太平。”四方百姓,裹黄巾从张角反者四五十万。
贼势浩大,官军望风而靡。何进奏帝火速降诏,令各处备御,讨贼立功。一面遣中郎将卢
植、皇甫嵩、朱儁,各引精兵、分三路讨之。

……

\chapter{张翼德怒鞭督邮 \ 何国舅谋诛宦竖}

且说董卓字仲颖,陇西临洮人也,官拜河东太守,自来骄傲。当日怠慢了玄德,张飞性发,便欲杀之。玄德与关公急止之曰;“他是朝廷命官,岂可擅杀?”飞曰:“若不杀这厮,反要在他部下听令,其实不甘!二兄要便住在此,我自投别处去也!”玄德曰:“我三人义同生死,岂可相离?不若都投别处去便了。”飞曰:“若如此,稍解吾恨。”

于是三人连夜引军来投朱俊。俊待之甚厚,合兵一处,进讨张宝。是时曹操自跟皇甫嵩讨张梁,大战于曲阳。这里朱俊进攻张宝。张宝引贼众八九万,屯于山后。俊令玄德为其先锋,与贼对敌。张宝遣副将高升出马搦战,玄德使张飞击之。飞纵马挺矛,与升交战,不数合,刺升落马。玄德麾军直冲过去。张宝就马上披发仗剑,作起妖法。只见风雷大作,一股黑气,从天而降,黑气中似有无限人马杀来。玄德连忙回军,军中大乱。败阵而归,与朱俊计议。俊曰:“彼用妖术,我来日可宰猪羊狗血,令军士伏于山头;候贼赶来,从高坡上泼之,其法可解。”玄德听令,拨关公、张飞各引军一千,伏于山后高冈之上,盛猪羊狗血并秽物准备。次日,张宝摇旗擂鼓,引军搦战,玄德出迎。交锋之际,张宝作法,风雷大作,飞砂走石,黑气漫天,滚滚人马,自天而下。玄德拨马便走,张宝驱兵赶来。将过山头,关、张伏军放起号炮,秽物齐泼。但见空中纸人草马,纷纷坠地;风雷顿息,砂石不飞。

张宝见解了法,急欲退军。左关公,右张飞,两军都出,背后玄德、朱俊一齐赶上,贼兵大败。玄德望见“地公将军”旗号,飞马赶来,张宝落荒而走。玄德发箭,中其左臂。张宝带箭逃脱,走入阳城,坚守不出。

……

\chapter{文档环境}

软件版本:

\begin{lstlisting}
PS C:\Users\Cheng> xelatex --version
XeTeX 3.14159265-2.6-0.99999 (TeX Live 2018/W32TeX)
\end{lstlisting}

本文档编译命令:

\begin{lstlisting}
xelatex.exe -synctex=1 -interaction=nonstopmode  -output-driver="xdvipdfmx -q -E -V 7" "kindle-pdf".tex
\end{lstlisting}

\end{document}